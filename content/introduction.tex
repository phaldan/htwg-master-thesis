\chapter{Einleitung}

In diesem Kapitel geht es im Rahmen einer Einleitung um die Heranführung an das Thema dieser Arbeit. Die Einleitung umfasst eine grobe Beschreibung der Problemstellung, eine Unterscheidung der verschiedenen wissenschaftlichen Fragestellungen die sich aus der Problemstellung ergeben und beschreibt den weiteren Verlauf dieser Arbeit. 

\section{Problemstellung}

Im Rahmen der unveröffentlichten Dissertation \textit{Generische Werkzeuge für domänenspezifische Grafische Sprachen} von Markus Gerhart ist unter der Leitung von Markus Boger das Projekt \textit{Zeta}, ehemals \textit{MoDiGen} genannt, an der Hochschule Konstanz Technik, Wirtschaft und Gestaltung entstanden. \textit{Zeta} ermöglicht zum einen die Erstellung einer eigenen graphischen \ac{dsl} und den entsprechenden Generatoren über einen Web Browser \cite{zeta_paper,zeta_masterthesis_actor}. Auf Basis einer graphischen \ac{dsl} generiert \textit{Zeta} einen graphischen Editor mit dem Modelle erstellt werden können. \textit{Zeta} hat als Proof of Concept angefangen und entwickelt sich inzwischen immer stärker zu einem eigenständigen Produkt. Die ursprüngliche Basis von \textit{Zeta} ist durch eine Vielzahl von Teamprojekten und Abschlussarbeiten stetig weiterentwickelt worden.

In diesem Rahmen soll \textit{Zeta} unter der Fragestellung \textit{Wie kann die Architektur für ein webbasiertes graphisches Modellierungsframework aussehen?} betrachtet werden. Zu diesem Zweck soll zu Beginn die Ausgangsarchitektur von \textit{Zeta} beschrieben und anhand von Best Practices und Entwurfsmustern bewertet werden. Abschließend sollen durch die im Rahmen dieser Arbeit angewendeten Optimierungen als Zielarchitektur vorgestellt werden.

\section{Aufbau}

Das nachfolgende Kapitel behandelt die Grundlagen. Bei den Grundlagen wird auf die verschiedenen Themen zum Verständnis dieser Arbeit eingegangen. Dies umfasst die Softwarearchitektur, die modellgetriebene Softwareentwicklung, die Webapplikation und die Containervirtualisierung. Auf die Grundlagen folgt das Kapitel über die Ausgangsarchitektur. Bei der Ausgangsarchitekur wird auf den Ausgangszustand der Architektur von \textit{Zeta} eingegangen und grob zwischen Front- und Back-End unterschieden. Nach der Ausgangsarchitektur folgt das Kapitel Review und betrachtet die Lösungen der Ausgangsarchitektur im Vergleich zu Best Practices und Entwurfsmustern. Auf das Review folgt das Kapitel Zielarchitekur und beschreibt die Architektur von \textit{Zeta} im Zielzustand mit den im Rahmen dieser Arbeit durchgeführten Anpassungen. Abschlossen wird diese Arbeit mit einem Resümee. Das Resümee enthält eine finale Schlussfolgerung und einen Ausblick auf zukünftige Arbeiten an der Architektur von \textit{Zeta}.

