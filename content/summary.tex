\chapter{Resümee}

Dieses Kapitel stellt den Abschluss dieser Arbeit dar und befasst sich zuerst mit einer finalen Schlussfolgerung über die Architektur von \textit{Zeta}. Als nächstes und letztes befasst sich dieses Kapitel mit einem Ausblick auf zukünftige Themen im Bereich der Architektur von \textit{Zeta}. 

\section{Schlussfolgerungen}

Das Review der Architektur im Ausgangszustand sollte hinreichend aufgezeigt haben, dass \textit{Zeta} noch ein hohes Optimierungspotenzial enthält. Auf allen Ebenen von der Laufzeitumgebung über die Kommunikation zwischen den einzelnen Diensten oder auch den Schichten innerhalb der Dienste, bis zur Qualität der Implementierung können noch erhebliche Verbesserungen bei \textit{Zeta} durchgeführt werden. \textit{Zeta} enthält viele interessante Ideen, aber die Umsetzungen dieser Ideen sind in vielen Fällen nur ausreichend bis mangelhaft. Das z.B. ein derart komplexes Thema wie der Parser für drei aufeinander aufbauenden \acp{dsl} nicht mal einen einzigen Unittest bekommen hat, bewerte ich als Kapitulation vor der Herausforderung der Implementierung eines Parsers für eine \ac{dsl}. Dazu kommt noch, dass der Parser im Ausgangszustand nicht mal in der Lage war bei einem Syntaxfehler in den Definitionen festzustellen in welchem der drei verschiedenen \acp{dsl} der Fehler vorhanden ist.

Im Rahmen dieser Arbeit wurde versucht für den Zielzustand eine solide Basis für die zukünftige Weiterentwicklung von \textit{Zeta} zu bieten. Dabei wurde versucht, solide Grenzen zwischen Front- und Back-End, zwischen den Schichten, zwischen den Diensten und zwischen den Module zu ziehen. Dabei sind die Grenzen in einer solchen Form gewählt, dass diese nicht properitäre Lösungen forcieren, sondern Offenheit für möglichst unterschiedliche Lösungen bieten. Aufgrund des Umfangs der bestehenden Umsetzung für \textit{Zeta} und der einhergehenden zeitlichen Limitierung dieser Arbeit wurde versucht die gröbsten Mängel in der Architektur zu lösen und die Mängel mit dem größtem Aufwand dem parallel laufenden Teamprojekt zu überlassen. Gleichzeitig sollte anhand der Einführung der Linter den Mitgliedern der aktuellen und zukünftigen Teamprojekte die Möglichkeit gegeben werden, Technische Schulden durch qualitativ niedrige Implementierung frühzeitig zu erkennen. Alles in allem bin ich mit dem Ergebnis dieser Arbeit zufrieden und bin davon überzeugt, dass diese Arbeit einen nachhaltig positiven Effekt auf die Weiterentwicklung von \textit{Zeta} hat.

\section{Ausblick}

Die Analyse und das Review der Architektur bringen in naher Zukunft schon die ersten Früchte beim parallel zu dieser Arbeit laufenden Teamprojekt. Der Parser für die textuellen \acp{dsl} ist im Rahmen einer Überarbeitung der \acp{dsl} vollständig neu implementiert worden. Dabei besteht dieser Parser auf mehreren aufeinander aufbauenden Teilschritten. Des Weiteren ist der Parser testgetrieben entwickelt worden. Außerdem ist der Parser nicht mehr Teil des \textit{Play Server}, sondern ist in ein eigenenes \ac{sbt} Unterprojekte ausgelagert worden. Eine erste frühe Version des Parsers wurde zum Abschluss dieser Arbeit in einem Pull Request zu \textit{Zeta} hinzugefügt \cite{zeta_pull_parser}.

Das Konzept hinter der Ausführung der Generatoren mit den Transformationsregeln müsste grundlegend überdacht werden. Der derzeitige Ansatz beeinflusst mit einem unkontrollierbaren Zugriff auf die Datenbank innerhalb der von Benutzer erstellten Transformationsregeln die Integrität des gesamten Systems. Desweiteren vereinen die Generatoren mit der Erstellung eines Eintrags des Generators in der Datenbank, der auch ein Beispiel für Transformationsregeln enthält, und der Laufzeitumgebung für die Transformationsregeln zwei voneinander funktional unabhängige Aufgaben. Ein \textit{Docker Container} ist zwar relativ leichtgewichtig und kann je nach Größe und Umfang, wie im Anhang~\ref{subsec:APPENDIX_LISTINGS_DOCKER_HELLO_WORLD} auf Seite~\pageref{subsec:APPENDIX_LISTINGS_DOCKER_HELLO_WORLD} zusehen, in unter einer Sekunde ausgeführt werden. Im Vergleich zu einem Actor liegen jedoch Welten zwischen dem Overhead eines \textit{Docker Containers} und dem eines Actors. Aus diesem Grund sollten Funktionalitäten, bei denen die Dauer der Ausführung nicht absehbar ist und eine spezielle Isolierung vom Rest des Systems benötigen, die Ausführung in einem externen \textit{Docker Container} stattfinden. Beides trifft nicht auf Funktionalität zur Erstellung der Generatoren in der Datenbank zu und auch nicht z.B. auf das Auslesen eines Eintrags, das geringe Modifizieren dieser Daten und das abschließende Erstellen eines neuen Eintrags in der Datenbank innerhalb des \textit{MetaModelRelease} Unterprojektes. Beides kann schon zuvor über einen Actor im Akka Cluster ausgeführt werden. Ohne dabei die Integrität von \textit{Zeta} negativ zu beeinflussen.

Das \textit{server} Unterprojekt des \textit{Play Servers} ist, als ehemals einzige Komponente von \textit{Zeta}, das mit Abstand größte Unterprojekt innerhalb des \ac{sbt} Projekts. Langfristig sollte der größte Teil der Implementierung von \textit{Zeta} aus dem \textit{Play Server} extrahiert werden und der \textit{Play Server} sollte nur noch als dünne oberste Schicht von \textit{Zeta} fungieren. Die Implementierung könnte in weitere \ac{sbt} Unterprojekte oder alternativ in den Akka Cluster wandern. Mit dem Akka Cluster Ansatz wäre auch eine bessere Skalierbarkeit gegeben und es ermöglicht zudem z.B. den parallelen Betrieb von mehreren \textit{Play Server} Instanzen. Das Ziel sollte darin bestehen, \textit{Zeta} weitestgehend vom \textit{Play Framework} zu isolieren. Um im Idealfall das \textit{Play Framework} als öffentliche \ac{api} austauschbar zu halten.

